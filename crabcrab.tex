% Options for packages loaded elsewhere
\PassOptionsToPackage{unicode}{hyperref}
\PassOptionsToPackage{hyphens}{url}
%
\documentclass[
]{article}
\title{Crab Research Project}
\author{}
\date{\vspace{-2.5em}}

\usepackage{amsmath,amssymb}
\usepackage{lmodern}
\usepackage{iftex}
\ifPDFTeX
  \usepackage[T1]{fontenc}
  \usepackage[utf8]{inputenc}
  \usepackage{textcomp} % provide euro and other symbols
\else % if luatex or xetex
  \usepackage{unicode-math}
  \defaultfontfeatures{Scale=MatchLowercase}
  \defaultfontfeatures[\rmfamily]{Ligatures=TeX,Scale=1}
\fi
% Use upquote if available, for straight quotes in verbatim environments
\IfFileExists{upquote.sty}{\usepackage{upquote}}{}
\IfFileExists{microtype.sty}{% use microtype if available
  \usepackage[]{microtype}
  \UseMicrotypeSet[protrusion]{basicmath} % disable protrusion for tt fonts
}{}
\makeatletter
\@ifundefined{KOMAClassName}{% if non-KOMA class
  \IfFileExists{parskip.sty}{%
    \usepackage{parskip}
  }{% else
    \setlength{\parindent}{0pt}
    \setlength{\parskip}{6pt plus 2pt minus 1pt}}
}{% if KOMA class
  \KOMAoptions{parskip=half}}
\makeatother
\usepackage{xcolor}
\IfFileExists{xurl.sty}{\usepackage{xurl}}{} % add URL line breaks if available
\IfFileExists{bookmark.sty}{\usepackage{bookmark}}{\usepackage{hyperref}}
\hypersetup{
  pdftitle={Crab Research Project},
  hidelinks,
  pdfcreator={LaTeX via pandoc}}
\urlstyle{same} % disable monospaced font for URLs
\usepackage[margin=1in]{geometry}
\usepackage{color}
\usepackage{fancyvrb}
\newcommand{\VerbBar}{|}
\newcommand{\VERB}{\Verb[commandchars=\\\{\}]}
\DefineVerbatimEnvironment{Highlighting}{Verbatim}{commandchars=\\\{\}}
% Add ',fontsize=\small' for more characters per line
\usepackage{framed}
\definecolor{shadecolor}{RGB}{248,248,248}
\newenvironment{Shaded}{\begin{snugshade}}{\end{snugshade}}
\newcommand{\AlertTok}[1]{\textcolor[rgb]{0.94,0.16,0.16}{#1}}
\newcommand{\AnnotationTok}[1]{\textcolor[rgb]{0.56,0.35,0.01}{\textbf{\textit{#1}}}}
\newcommand{\AttributeTok}[1]{\textcolor[rgb]{0.77,0.63,0.00}{#1}}
\newcommand{\BaseNTok}[1]{\textcolor[rgb]{0.00,0.00,0.81}{#1}}
\newcommand{\BuiltInTok}[1]{#1}
\newcommand{\CharTok}[1]{\textcolor[rgb]{0.31,0.60,0.02}{#1}}
\newcommand{\CommentTok}[1]{\textcolor[rgb]{0.56,0.35,0.01}{\textit{#1}}}
\newcommand{\CommentVarTok}[1]{\textcolor[rgb]{0.56,0.35,0.01}{\textbf{\textit{#1}}}}
\newcommand{\ConstantTok}[1]{\textcolor[rgb]{0.00,0.00,0.00}{#1}}
\newcommand{\ControlFlowTok}[1]{\textcolor[rgb]{0.13,0.29,0.53}{\textbf{#1}}}
\newcommand{\DataTypeTok}[1]{\textcolor[rgb]{0.13,0.29,0.53}{#1}}
\newcommand{\DecValTok}[1]{\textcolor[rgb]{0.00,0.00,0.81}{#1}}
\newcommand{\DocumentationTok}[1]{\textcolor[rgb]{0.56,0.35,0.01}{\textbf{\textit{#1}}}}
\newcommand{\ErrorTok}[1]{\textcolor[rgb]{0.64,0.00,0.00}{\textbf{#1}}}
\newcommand{\ExtensionTok}[1]{#1}
\newcommand{\FloatTok}[1]{\textcolor[rgb]{0.00,0.00,0.81}{#1}}
\newcommand{\FunctionTok}[1]{\textcolor[rgb]{0.00,0.00,0.00}{#1}}
\newcommand{\ImportTok}[1]{#1}
\newcommand{\InformationTok}[1]{\textcolor[rgb]{0.56,0.35,0.01}{\textbf{\textit{#1}}}}
\newcommand{\KeywordTok}[1]{\textcolor[rgb]{0.13,0.29,0.53}{\textbf{#1}}}
\newcommand{\NormalTok}[1]{#1}
\newcommand{\OperatorTok}[1]{\textcolor[rgb]{0.81,0.36,0.00}{\textbf{#1}}}
\newcommand{\OtherTok}[1]{\textcolor[rgb]{0.56,0.35,0.01}{#1}}
\newcommand{\PreprocessorTok}[1]{\textcolor[rgb]{0.56,0.35,0.01}{\textit{#1}}}
\newcommand{\RegionMarkerTok}[1]{#1}
\newcommand{\SpecialCharTok}[1]{\textcolor[rgb]{0.00,0.00,0.00}{#1}}
\newcommand{\SpecialStringTok}[1]{\textcolor[rgb]{0.31,0.60,0.02}{#1}}
\newcommand{\StringTok}[1]{\textcolor[rgb]{0.31,0.60,0.02}{#1}}
\newcommand{\VariableTok}[1]{\textcolor[rgb]{0.00,0.00,0.00}{#1}}
\newcommand{\VerbatimStringTok}[1]{\textcolor[rgb]{0.31,0.60,0.02}{#1}}
\newcommand{\WarningTok}[1]{\textcolor[rgb]{0.56,0.35,0.01}{\textbf{\textit{#1}}}}
\usepackage{graphicx}
\makeatletter
\def\maxwidth{\ifdim\Gin@nat@width>\linewidth\linewidth\else\Gin@nat@width\fi}
\def\maxheight{\ifdim\Gin@nat@height>\textheight\textheight\else\Gin@nat@height\fi}
\makeatother
% Scale images if necessary, so that they will not overflow the page
% margins by default, and it is still possible to overwrite the defaults
% using explicit options in \includegraphics[width, height, ...]{}
\setkeys{Gin}{width=\maxwidth,height=\maxheight,keepaspectratio}
% Set default figure placement to htbp
\makeatletter
\def\fps@figure{htbp}
\makeatother
\setlength{\emergencystretch}{3em} % prevent overfull lines
\providecommand{\tightlist}{%
  \setlength{\itemsep}{0pt}\setlength{\parskip}{0pt}}
\setcounter{secnumdepth}{-\maxdimen} % remove section numbering
\ifLuaTeX
  \usepackage{selnolig}  % disable illegal ligatures
\fi

\begin{document}
\maketitle

\hypertarget{team}{%
\subsubsection{Team:}\label{team}}

\begin{itemize}
\tightlist
\item
  Alina Muliak
\item
  Olexiy Hoiev
\item
  Oleksandra Stasiuk
\end{itemize}

\hypertarget{introduction}{%
\subsubsection{Introduction}\label{introduction}}

Crab farming is practiced worldwide as true crabs make up 20\% of all
crustaceans caught and farmed in the world, with about 1.4 million
tonnes being consumed annually. But there are some restrictions and
challenges in profit based on when the crabs should be farmed. For a
commercial crab farmer knowing the right age of the crab helps them
decide if and when to harvest the crabs. Beyond a certain age, there is
negligible growth in crab's physical characteristics and hence, it is
important to time the harvesting to reduce cost and increase profit.

\hypertarget{aim}{%
\subsubsection{Aim}\label{aim}}

The main purpose of this research is to provide farmers with information
on age of the crabs based on their physical attributes, to help them
decide on the most suitable time to harvest the crabs. Moreover, the
goal is also to make this decision as easy and realistic as possible,
relying on the minimum parameters getting accurate prediction.

\hypertarget{analysis-of-the-data}{%
\subsubsection{Analysis of the data}\label{analysis-of-the-data}}

The main data set for the project consist of crab's physical
characteristics, farmed in Boston area, that include sex, length,
diameter, height, age, weight, shucked weight, viscera weight and shell
weight.

First, we download necessary library and set seed to easier work with
results.

\begin{Shaded}
\begin{Highlighting}[]
\FunctionTok{library}\NormalTok{(ggplot2)}
\FunctionTok{set.seed}\NormalTok{(}\DecValTok{125368}\NormalTok{)}
\end{Highlighting}
\end{Shaded}

To prepare our data, as our goal is to help farmers harvest crabs, we
leave out only those parameters, which farmers can estimate without
killing the crabs: sex, length, diameter, height, weight and age.

The graphs below show the dependence on every parameter in relation to
others. This information helps us visually see which parameters will
help to make an accurate prediction.

\begin{Shaded}
\begin{Highlighting}[]
\NormalTok{data }\OtherTok{\textless{}{-}} \FunctionTok{read.csv}\NormalTok{(}\StringTok{"./CrabAgePrediction.csv"}\NormalTok{, }\ConstantTok{TRUE}\NormalTok{)}

\NormalTok{data }\OtherTok{\textless{}{-}}\NormalTok{ data[,}\SpecialCharTok{{-}}\DecValTok{6}\SpecialCharTok{:{-}}\DecValTok{8}\NormalTok{]}
\NormalTok{data.sorted }\OtherTok{\textless{}{-}}\NormalTok{ data[}\FunctionTok{order}\NormalTok{(data}\SpecialCharTok{$}\NormalTok{Age),]}
\NormalTok{data.shuffled }\OtherTok{\textless{}{-}}\NormalTok{ data[}\FunctionTok{sample}\NormalTok{(}\DecValTok{1}\SpecialCharTok{:}\FunctionTok{nrow}\NormalTok{(data)), ]}
\CommentTok{\# data$Length[1:100]}
\FunctionTok{plot}\NormalTok{(data, }\AttributeTok{col=}\StringTok{"purple"}\NormalTok{)}
\end{Highlighting}
\end{Shaded}

\includegraphics{crabcrab_files/figure-latex/unnamed-chunk-2-1.pdf} We
can see that all the parameters' graphs shows strong relations, but on
the height's graph there is almost no changes in dependence to others.

\begin{Shaded}
\begin{Highlighting}[]
\FunctionTok{plot}\NormalTok{(data}\SpecialCharTok{$}\NormalTok{Height, }\AttributeTok{col=}\StringTok{"purple"}\NormalTok{)}
\end{Highlighting}
\end{Shaded}

\includegraphics{crabcrab_files/figure-latex/unnamed-chunk-3-1.pdf}

To further prepare the data, we split it in 2 parts. First one we will
need for building a linear regression model, and second will be for
testing. As they are different, it will assure us of a better accuracy
in our prediction.

\begin{Shaded}
\begin{Highlighting}[]
\NormalTok{data.train }\OtherTok{\textless{}{-}}\NormalTok{ data.shuffled[}\DecValTok{1}\SpecialCharTok{:}\DecValTok{2000}\NormalTok{,]}
\NormalTok{data.train }\OtherTok{\textless{}{-}}\NormalTok{ data.train[}\FunctionTok{order}\NormalTok{(data.train}\SpecialCharTok{$}\NormalTok{Age),]}
\NormalTok{data.test }\OtherTok{\textless{}{-}}\NormalTok{ data.shuffled[}\DecValTok{2001}\SpecialCharTok{:}\DecValTok{4000}\NormalTok{,]}
\NormalTok{data.test }\OtherTok{\textless{}{-}}\NormalTok{ data.test[}\FunctionTok{order}\NormalTok{(data.test}\SpecialCharTok{$}\NormalTok{Age),]}

\NormalTok{data.train[}\DecValTok{1}\SpecialCharTok{:}\DecValTok{20}\NormalTok{,]}
\end{Highlighting}
\end{Shaded}

\begin{verbatim}
##      Sex Length Diameter Height    Weight Age
## 1352   M 0.5250   0.3750 0.1250 1.0914557   3
## 2094   I 0.6000   0.4375 0.1625 1.8852418   3
## 2845   I 0.5125   0.3875 0.1125 1.4033003   3
## 754    I 0.2750   0.2250 0.0750 0.2267960   3
## 1222   I 0.5375   0.3750 0.1375 1.1623295   3
## 3456   I 0.4500   0.3250 0.1125 0.7796112   3
## 2766   M 0.3875   0.2750 0.1000 0.4394172   3
## 1169   I 0.4125   0.3000 0.1250 0.5953395   3
## 308    I 0.3500   0.2625 0.0875 0.3968930   3
## 1342   I 0.6000   0.4500 0.1375 1.5733972   4
## 1133   M 0.6125   0.4500 0.1625 1.8001932   4
## 929    I 0.5500   0.4000 0.1250 1.3891255   4
## 2913   I 0.4250   0.2625 0.0875 0.9638830   4
## 2639   M 0.5875   0.4000 0.1500 1.5450477   4
## 948    I 0.5250   0.3750 0.1125 1.1339800   4
## 3548   I 0.5625   0.4250 0.1750 1.6017467   4
## 1233   I 0.6125   0.4750 0.1500 2.4380570   4
## 1039   I 0.4625   0.3250 0.1125 0.8221355   4
## 1124   I 0.4125   0.2875 0.0375 0.4110678   4
## 2734   I 0.7875   0.5875 0.1875 3.6429107   4
\end{verbatim}

\begin{Shaded}
\begin{Highlighting}[]
\NormalTok{data.test[}\DecValTok{1}\SpecialCharTok{:}\DecValTok{20}\NormalTok{,]}
\end{Highlighting}
\end{Shaded}

\begin{verbatim}
##      Sex Length Diameter Height    Weight Age
## 1331   I 0.1875   0.1375 0.0250 0.0566990   1
## 1135   I 0.3750   0.2500 0.0625 0.4252425   2
## 600    I 0.4875   0.3750 0.1125 1.0631062   3
## 3444   I 0.4750   0.3500 0.0750 0.8930092   3
## 2779   I 0.4000   0.2750 0.0625 0.5102910   3
## 3729   I 0.3250   0.2500 0.0750 0.3685435   3
## 2865   I 0.6125   0.4875 0.1500 2.6932025   4
## 1943   I 0.6250   0.4500 0.1625 2.2821347   4
## 1648   I 0.4000   0.2750 0.0625 0.5528152   4
## 3672   I 0.4250   0.3250 0.2375 0.8504850   4
## 956    I 0.3500   0.2625 0.0875 0.4110678   4
## 1290   M 0.4875   0.3625 0.1250 0.9071840   4
## 1593   I 0.5750   0.4125 0.1500 1.4599993   4
## 1901   I 0.3375   0.3250 0.1000 0.8221355   4
## 1808   I 0.5000   0.3625 0.1500 1.0489315   4
## 697    I 0.5000   0.3625 0.1250 1.0205820   4
## 1717   I 0.6375   0.4500 0.1375 2.3530085   4
## 3063   I 0.6750   0.5125 0.1875 3.3452410   4
## 3567   I 0.5625   0.4250 0.1250 1.4599993   4
## 2300   M 0.8125   0.5750 0.2250 4.1673765   4
\end{verbatim}

The data set is not ideal, and in the parameter ``sex'' we have three
categories - M stands for male, F - for female, and I - indeterminate.
Below we build two graphs on parameters length and age in relation to
different sexes.

\begin{Shaded}
\begin{Highlighting}[]
\FunctionTok{ggplot}\NormalTok{(data) }\SpecialCharTok{+}
    \FunctionTok{geom\_density}\NormalTok{(}\FunctionTok{aes}\NormalTok{(}\AttributeTok{x =}\NormalTok{ Length, }\AttributeTok{fill =}\NormalTok{ Sex), }\AttributeTok{alpha =} \FloatTok{0.5}\NormalTok{) }\SpecialCharTok{+}
    \FunctionTok{labs}\NormalTok{(}\AttributeTok{y =} \StringTok{"Density"}\NormalTok{, }\AttributeTok{x =} \StringTok{"Length"}\NormalTok{, }\AttributeTok{fill =} \StringTok{"Crabs\textquotesingle{} Sex"}\NormalTok{) }\SpecialCharTok{+}
    \FunctionTok{scale\_fill\_manual}\NormalTok{(}\AttributeTok{labels =} \FunctionTok{c}\NormalTok{(}\StringTok{"M"}\NormalTok{, }\StringTok{"F"}\NormalTok{, }\StringTok{"I"}\NormalTok{),}
        \AttributeTok{values =} \FunctionTok{c}\NormalTok{(}\StringTok{"\#DB24BC"}\NormalTok{, }\StringTok{"\#BCDB24"}\NormalTok{, }\StringTok{"\#24BCDB"}\NormalTok{)) }\SpecialCharTok{+}
    \FunctionTok{theme\_minimal}\NormalTok{()}
\end{Highlighting}
\end{Shaded}

\includegraphics{crabcrab_files/figure-latex/unnamed-chunk-5-1.pdf}

\begin{Shaded}
\begin{Highlighting}[]
\FunctionTok{ggplot}\NormalTok{(data) }\SpecialCharTok{+}
    \FunctionTok{geom\_density}\NormalTok{(}\FunctionTok{aes}\NormalTok{(}\AttributeTok{x =}\NormalTok{ Age, }\AttributeTok{fill =}\NormalTok{ Sex), }\AttributeTok{alpha =} \FloatTok{0.6}\NormalTok{) }\SpecialCharTok{+}
    \FunctionTok{labs}\NormalTok{(}\AttributeTok{y =} \StringTok{"Density"}\NormalTok{, }\AttributeTok{x =} \StringTok{"Age"}\NormalTok{, }\AttributeTok{fill =} \StringTok{"Crabs\textquotesingle{} Sex"}\NormalTok{) }\SpecialCharTok{+}
    \FunctionTok{scale\_fill\_manual}\NormalTok{(}\AttributeTok{labels =} \FunctionTok{c}\NormalTok{(}\StringTok{"M"}\NormalTok{, }\StringTok{"F"}\NormalTok{, }\StringTok{"I"}\NormalTok{),}
        \AttributeTok{values =} \FunctionTok{c}\NormalTok{(}\StringTok{"\#DB24BC"}\NormalTok{, }\StringTok{"\#BCDB24"}\NormalTok{, }\StringTok{"\#24BCDB"}\NormalTok{)) }\SpecialCharTok{+}
    \FunctionTok{theme\_minimal}\NormalTok{()}
\end{Highlighting}
\end{Shaded}

\includegraphics{crabcrab_files/figure-latex/unnamed-chunk-5-2.pdf}

\begin{Shaded}
\begin{Highlighting}[]
\FunctionTok{head}\NormalTok{(data)}
\end{Highlighting}
\end{Shaded}

\begin{verbatim}
##   Sex Length Diameter Height    Weight Age
## 1   F 1.4375   1.1750 0.4125 24.635715   9
## 2   M 0.8875   0.6500 0.2125  5.400580   6
## 3   I 1.0375   0.7750 0.2500  7.952035   6
## 4   F 1.1750   0.8875 0.2500 13.480187  10
## 5   I 0.8875   0.6625 0.2125  6.903103   6
## 6   F 1.5500   1.1625 0.3500 28.661344   8
\end{verbatim}

The female category is very different in both parameters from other two,
but indeterminate is almost the same as male, which make us believe that
farmers couldn't determine the sex of mainly male crabs.

\hypertarget{anova-test}{%
\subsubsection{ANOVA Test}\label{anova-test}}

For building the prediction, we first need ANOVA test. \textbf{ANOVA},
which stands for Analysis of Variance, is an extension of a \(t\)-test.
It tells if there are any statistical differences between the means of
three or more independent groups. Like the t-test, ANOVA helps to find
out whether the differences between groups of data are statistically
significant. It works by analyzing the levels of variance within the
groups through samples taken from each of them.

\(F\)-value:
\[F = \frac{\text{Between group variatioin}}{\text{Within group variation}}\]
If most of the variation comes from within groups, then `some component'
probably does not have much of any effect, however, if the variation
seems most to come from between the groups, this would indicate that
`some component' does probably have an effect.

Thus, larger \(F\)-ratios indicate there's a high probability that the
groups do have different means; in other words, it does make a
difference. High \(f\)-ratio combined with a small \(p\)-value means
that `some component' most likely does have an effect and we should
reject the null hypothesis that the means are the same.

We build different linear regression models. As our goal is to minimize
the use of parameters for farmers to decide easily, we compare linear
model with all parameters and some without height, length and diameter.
It will help us to see which one we can ignore while getting relatively
accurate results.

\begin{enumerate}
\def\labelenumi{\arabic{enumi}.}
\tightlist
\item
  First one considers all the parameters we're have in prepared data
  set.
\end{enumerate}

\begin{Shaded}
\begin{Highlighting}[]
\NormalTok{lm.all }\OtherTok{\textless{}{-}} \FunctionTok{lm}\NormalTok{(Age }\SpecialCharTok{\textasciitilde{}}\NormalTok{ Weight }\SpecialCharTok{+}\NormalTok{ Diameter }\SpecialCharTok{+}\NormalTok{ Sex }\SpecialCharTok{+}\NormalTok{ Length }\SpecialCharTok{+}\NormalTok{ Height, }\AttributeTok{data =}\NormalTok{ data.train)}
\NormalTok{fit }\OtherTok{\textless{}{-}} \FunctionTok{aov}\NormalTok{(lm.all, }\AttributeTok{data=}\NormalTok{data)}
\FunctionTok{summary}\NormalTok{(fit)}
\end{Highlighting}
\end{Shaded}

\begin{verbatim}
##               Df Sum Sq Mean Sq F value   Pr(>F)    
## Weight         1  11723   11723 1787.35  < 2e-16 ***
## Diameter       1   1590    1590  242.39  < 2e-16 ***
## Sex            2    780     390   59.45  < 2e-16 ***
## Length         1    171     171   26.07 3.46e-07 ***
## Height         1    627     627   95.63  < 2e-16 ***
## Residuals   3886  25487       7                     
## ---
## Signif. codes:  0 '***' 0.001 '**' 0.01 '*' 0.05 '.' 0.1 ' ' 1
\end{verbatim}

As we can see in the summary, all attributes have large \(f\)-ratio
combined with very small \(p\)-values.

\begin{enumerate}
\def\labelenumi{\arabic{enumi}.}
\setcounter{enumi}{1}
\tightlist
\item
  Second is now with all parameters except for height.
\end{enumerate}

\begin{Shaded}
\begin{Highlighting}[]
\NormalTok{lm.no\_high }\OtherTok{\textless{}{-}} \FunctionTok{lm}\NormalTok{(Age }\SpecialCharTok{\textasciitilde{}}\NormalTok{ Weight }\SpecialCharTok{+}\NormalTok{ Diameter }\SpecialCharTok{+}\NormalTok{ Sex }\SpecialCharTok{+}\NormalTok{ Length , }\AttributeTok{data =}\NormalTok{ data.train)}
\NormalTok{fit }\OtherTok{\textless{}{-}} \FunctionTok{aov}\NormalTok{(lm.no\_high, }\AttributeTok{data=}\NormalTok{data)}
\FunctionTok{summary}\NormalTok{(fit)}
\end{Highlighting}
\end{Shaded}

\begin{verbatim}
##               Df Sum Sq Mean Sq F value   Pr(>F)    
## Weight         1  11723   11723 1744.87  < 2e-16 ***
## Diameter       1   1590    1590  236.63  < 2e-16 ***
## Sex            2    780     390   58.04  < 2e-16 ***
## Length         1    171     171   25.45 4.76e-07 ***
## Residuals   3887  26115       7                     
## ---
## Signif. codes:  0 '***' 0.001 '**' 0.01 '*' 0.05 '.' 0.1 ' ' 1
\end{verbatim}

\begin{enumerate}
\def\labelenumi{\arabic{enumi}.}
\setcounter{enumi}{2}
\tightlist
\item
  Without length.
\end{enumerate}

\begin{Shaded}
\begin{Highlighting}[]
\NormalTok{lm.no\_len }\OtherTok{\textless{}{-}} \FunctionTok{lm}\NormalTok{(Age }\SpecialCharTok{\textasciitilde{}}\NormalTok{ Weight }\SpecialCharTok{+}\NormalTok{ Diameter }\SpecialCharTok{+}\NormalTok{ Sex }\SpecialCharTok{+}\NormalTok{ Height, }\AttributeTok{data =}\NormalTok{ data.train)}
\NormalTok{fit }\OtherTok{\textless{}{-}} \FunctionTok{aov}\NormalTok{(lm.no\_len, }\AttributeTok{data=}\NormalTok{data)}
\FunctionTok{summary}\NormalTok{(fit)}
\end{Highlighting}
\end{Shaded}

\begin{verbatim}
##               Df Sum Sq Mean Sq F value Pr(>F)    
## Weight         1  11723   11723 1775.24 <2e-16 ***
## Diameter       1   1590    1590  240.75 <2e-16 ***
## Sex            2    780     390   59.05 <2e-16 ***
## Height         1    618     618   93.56 <2e-16 ***
## Residuals   3887  25668       7                   
## ---
## Signif. codes:  0 '***' 0.001 '**' 0.01 '*' 0.05 '.' 0.1 ' ' 1
\end{verbatim}

\begin{enumerate}
\def\labelenumi{\arabic{enumi}.}
\setcounter{enumi}{3}
\tightlist
\item
  Without diameter.
\end{enumerate}

\begin{Shaded}
\begin{Highlighting}[]
\NormalTok{lm.no\_diameter }\OtherTok{\textless{}{-}} \FunctionTok{lm}\NormalTok{(Age }\SpecialCharTok{\textasciitilde{}}\NormalTok{ Weight }\SpecialCharTok{+}\NormalTok{ Sex }\SpecialCharTok{+}\NormalTok{ Length }\SpecialCharTok{+}\NormalTok{ Height, }\AttributeTok{data =}\NormalTok{ data.train)}
\NormalTok{fit }\OtherTok{\textless{}{-}} \FunctionTok{aov}\NormalTok{(lm.no\_diameter, }\AttributeTok{data=}\NormalTok{data)}
\FunctionTok{summary}\NormalTok{(fit)}
\end{Highlighting}
\end{Shaded}

\begin{verbatim}
##               Df Sum Sq Mean Sq F value Pr(>F)    
## Weight         1  11723   11723 1751.63 <2e-16 ***
## Sex            2   1169     585   87.36 <2e-16 ***
## Length         1    668     668   99.75 <2e-16 ***
## Height         1    805     805  120.21 <2e-16 ***
## Residuals   3887  26014       7                   
## ---
## Signif. codes:  0 '***' 0.001 '**' 0.01 '*' 0.05 '.' 0.1 ' ' 1
\end{verbatim}

\begin{enumerate}
\def\labelenumi{\arabic{enumi}.}
\setcounter{enumi}{4}
\tightlist
\item
  Without diameter and height.
\end{enumerate}

\begin{Shaded}
\begin{Highlighting}[]
\NormalTok{lm.no\_diameter\_height }\OtherTok{\textless{}{-}} \FunctionTok{lm}\NormalTok{(Age }\SpecialCharTok{\textasciitilde{}}\NormalTok{ Weight }\SpecialCharTok{+}\NormalTok{ Sex }\SpecialCharTok{+}\NormalTok{ Length, }\AttributeTok{data =}\NormalTok{ data.train)}
\NormalTok{fit }\OtherTok{\textless{}{-}} \FunctionTok{aov}\NormalTok{(lm.no\_diameter\_height, }\AttributeTok{data=}\NormalTok{data)}
\FunctionTok{summary}\NormalTok{(fit)}
\end{Highlighting}
\end{Shaded}

\begin{verbatim}
##               Df Sum Sq Mean Sq F value Pr(>F)    
## Weight         1  11723   11723 1699.52 <2e-16 ***
## Sex            2   1169     585   84.76 <2e-16 ***
## Length         1    668     668   96.78 <2e-16 ***
## Residuals   3888  26818       7                   
## ---
## Signif. codes:  0 '***' 0.001 '**' 0.01 '*' 0.05 '.' 0.1 ' ' 1
\end{verbatim}

\hypertarget{testing-linear-models}{%
\subsubsection{Testing linear models}\label{testing-linear-models}}

Now we are going to test our regression models on real data, which we
reserved for testing. With that we will see how great is the change if
we ignore some of parameters, and which ones are more important.

First, we make function to calculate values based on our regression
model coefficients.

\begin{Shaded}
\begin{Highlighting}[]
\NormalTok{calculate\_value }\OtherTok{\textless{}{-}} \ControlFlowTok{function}\NormalTok{(a, }\AttributeTok{weight=}\DecValTok{0}\NormalTok{, }\AttributeTok{diameter=}\DecValTok{0}\NormalTok{, }\AttributeTok{sexi=}\DecValTok{0}\NormalTok{, }\AttributeTok{sexm=}\DecValTok{0}\NormalTok{, }\AttributeTok{length=}\DecValTok{0}\NormalTok{, }\AttributeTok{height=}\DecValTok{0}\NormalTok{, values)\{}
\NormalTok{  res }\OtherTok{=}\NormalTok{ a }\SpecialCharTok{+}\NormalTok{ weight }\SpecialCharTok{*}\NormalTok{ values}\SpecialCharTok{$}\NormalTok{Weight }\SpecialCharTok{+}\NormalTok{ diameter }\SpecialCharTok{*}\NormalTok{ values}\SpecialCharTok{$}\NormalTok{Diameter }\SpecialCharTok{+}\NormalTok{ sexm }\SpecialCharTok{*}\NormalTok{ (values}\SpecialCharTok{$}\NormalTok{Sex }\SpecialCharTok{==} \StringTok{"M"}\NormalTok{)}\SpecialCharTok{*}\DecValTok{1} \SpecialCharTok{+}\NormalTok{ sexi }\SpecialCharTok{*}\NormalTok{ (values}\SpecialCharTok{$}\NormalTok{Sex }\SpecialCharTok{==} \StringTok{"I"}\NormalTok{)}\SpecialCharTok{*}\DecValTok{1} \SpecialCharTok{+}\NormalTok{ length }\SpecialCharTok{*}\NormalTok{ values}\SpecialCharTok{$}\NormalTok{Length }\SpecialCharTok{+}\NormalTok{ height}\SpecialCharTok{*}\NormalTok{values}\SpecialCharTok{$}\NormalTok{Height }
  \FunctionTok{return}\NormalTok{(res)}
\NormalTok{\}}
\end{Highlighting}
\end{Shaded}

And for comparing it to actual values.

\begin{Shaded}
\begin{Highlighting}[]
\NormalTok{compare\_to\_actual }\OtherTok{\textless{}{-}} \ControlFlowTok{function}\NormalTok{(prediction, data, param1, param2)\{}
\NormalTok{  prediction.bool }\OtherTok{\textless{}{-}}\NormalTok{ prediction }\SpecialCharTok{\textgreater{}}\NormalTok{ param1}
\NormalTok{  data.test.age.bool }\OtherTok{\textless{}{-}}\NormalTok{ data}\SpecialCharTok{$}\NormalTok{Age }\SpecialCharTok{\textgreater{}}\NormalTok{ param2}
  
\NormalTok{  res }\OtherTok{\textless{}{-}}\NormalTok{ (prediction.bool }\SpecialCharTok{==}\NormalTok{ data.test.age.bool)}
  
\NormalTok{  sum }\OtherTok{\textless{}{-}} \DecValTok{0}
  \ControlFlowTok{for}\NormalTok{(i }\ControlFlowTok{in} \FunctionTok{c}\NormalTok{(}\DecValTok{1}\SpecialCharTok{:}\DecValTok{1893}\NormalTok{))}
\NormalTok{    sum }\OtherTok{=}\NormalTok{ sum }\SpecialCharTok{+}\NormalTok{ res[i]}
  \FunctionTok{return}\NormalTok{(sum }\SpecialCharTok{/} \DecValTok{1893}\NormalTok{)}
\NormalTok{\}}
\end{Highlighting}
\end{Shaded}

Now we use them to see how good our linear regression model is with
includiong different parameters.

\begin{enumerate}
\def\labelenumi{\arabic{enumi}.}
\tightlist
\item
  First, we test the model with all parameters included.
\end{enumerate}

\begin{Shaded}
\begin{Highlighting}[]
\NormalTok{lm.coefficients }\OtherTok{\textless{}{-}}\NormalTok{ lm.all}\SpecialCharTok{$}\NormalTok{coefficients}

\NormalTok{prediction }\OtherTok{\textless{}{-}} \FunctionTok{calculate\_value}\NormalTok{(lm.coefficients[}\DecValTok{1}\NormalTok{], lm.coefficients[}\DecValTok{2}\NormalTok{], lm.coefficients[}\DecValTok{3}\NormalTok{], lm.coefficients[}\DecValTok{4}\NormalTok{], lm.coefficients[}\DecValTok{5}\NormalTok{], lm.coefficients[}\DecValTok{6}\NormalTok{], lm.coefficients[}\DecValTok{7}\NormalTok{], }\AttributeTok{values =}\NormalTok{ data.test)}
\FunctionTok{plot}\NormalTok{(}\FunctionTok{sort}\NormalTok{(prediction))}
\FunctionTok{lines}\NormalTok{(data.test}\SpecialCharTok{$}\NormalTok{Age, }\AttributeTok{col =} \StringTok{"red"}\NormalTok{, }\AttributeTok{lwd =} \DecValTok{5}\NormalTok{)}
\end{Highlighting}
\end{Shaded}

\includegraphics{crabcrab_files/figure-latex/unnamed-chunk-13-1.pdf}

\begin{Shaded}
\begin{Highlighting}[]
\FunctionTok{compare\_to\_actual}\NormalTok{(prediction, data.test, }\DecValTok{8}\NormalTok{, }\DecValTok{8}\NormalTok{)}
\end{Highlighting}
\end{Shaded}

\begin{verbatim}
## [1] 0.8066561
\end{verbatim}

This is our base result that we will compare to others.

\begin{enumerate}
\def\labelenumi{\arabic{enumi}.}
\setcounter{enumi}{1}
\tightlist
\item
  Test model without height parameter.
\end{enumerate}

\begin{Shaded}
\begin{Highlighting}[]
\NormalTok{lm.coefficients }\OtherTok{\textless{}{-}}\NormalTok{ lm.no\_high}\SpecialCharTok{$}\NormalTok{coefficients}

\NormalTok{prediction }\OtherTok{\textless{}{-}} \FunctionTok{calculate\_value}\NormalTok{(}\AttributeTok{a =}\NormalTok{ lm.coefficients[}\DecValTok{1}\NormalTok{], }\AttributeTok{weight =}\NormalTok{ lm.coefficients[}\DecValTok{2}\NormalTok{], }\AttributeTok{diameter =}\NormalTok{ lm.coefficients[}\DecValTok{3}\NormalTok{], }\AttributeTok{sexi =}\NormalTok{ lm.coefficients[}\DecValTok{4}\NormalTok{], }\AttributeTok{sexm =}\NormalTok{ lm.coefficients[}\DecValTok{5}\NormalTok{], }\AttributeTok{length =}\NormalTok{ lm.coefficients[}\DecValTok{6}\NormalTok{], }\AttributeTok{values =}\NormalTok{ data.test)}
\FunctionTok{plot}\NormalTok{(}\FunctionTok{sort}\NormalTok{(prediction))}
\FunctionTok{lines}\NormalTok{(data.test}\SpecialCharTok{$}\NormalTok{Age, }\AttributeTok{col =} \StringTok{"red"}\NormalTok{, }\AttributeTok{lwd =} \DecValTok{5}\NormalTok{)}
\end{Highlighting}
\end{Shaded}

\includegraphics{crabcrab_files/figure-latex/unnamed-chunk-14-1.pdf}

\begin{Shaded}
\begin{Highlighting}[]
\FunctionTok{compare\_to\_actual}\NormalTok{(prediction, data.test, }\DecValTok{8}\NormalTok{, }\DecValTok{8}\NormalTok{)}
\end{Highlighting}
\end{Shaded}

\begin{verbatim}
## [1] 0.8082409
\end{verbatim}

Here we can see that the result became a little worse but is still
precise, in comparison to our base model. It is also seen on the
prediction graph. Therefore, we can conclude that height is not that
important and can be ignored by farmers.

\begin{enumerate}
\def\labelenumi{\arabic{enumi}.}
\setcounter{enumi}{2}
\tightlist
\item
  Test model without length parameter.
\end{enumerate}

\begin{Shaded}
\begin{Highlighting}[]
\NormalTok{lm.coefficients }\OtherTok{\textless{}{-}}\NormalTok{ lm.no\_high}\SpecialCharTok{$}\NormalTok{coefficients}

\NormalTok{prediction }\OtherTok{\textless{}{-}} \FunctionTok{calculate\_value}\NormalTok{(}\AttributeTok{a =}\NormalTok{ lm.coefficients[}\DecValTok{1}\NormalTok{], }\AttributeTok{weight =}\NormalTok{ lm.coefficients[}\DecValTok{2}\NormalTok{], }\AttributeTok{diameter =}\NormalTok{ lm.coefficients[}\DecValTok{3}\NormalTok{], }\AttributeTok{sexi =}\NormalTok{ lm.coefficients[}\DecValTok{4}\NormalTok{], }\AttributeTok{sexm =}\NormalTok{ lm.coefficients[}\DecValTok{5}\NormalTok{], }\AttributeTok{height =}\NormalTok{ lm.coefficients[}\DecValTok{6}\NormalTok{], }\AttributeTok{values =}\NormalTok{ data.test)}
\FunctionTok{plot}\NormalTok{(}\FunctionTok{sort}\NormalTok{(prediction))}
\FunctionTok{lines}\NormalTok{(data.test}\SpecialCharTok{$}\NormalTok{Age, }\AttributeTok{col =} \StringTok{"red"}\NormalTok{, }\AttributeTok{lwd =} \DecValTok{5}\NormalTok{)}
\end{Highlighting}
\end{Shaded}

\includegraphics{crabcrab_files/figure-latex/unnamed-chunk-15-1.pdf}

\begin{Shaded}
\begin{Highlighting}[]
\FunctionTok{compare\_to\_actual}\NormalTok{(prediction, data.test, }\DecValTok{8}\NormalTok{, }\DecValTok{8}\NormalTok{)}
\end{Highlighting}
\end{Shaded}

\begin{verbatim}
## [1] 0.7152668
\end{verbatim}

Without length, our results became much worse, the value is different
and graphs are further from each other in predicted and actual values.
Therefore, we don't want to get rid of this parameter to get accurate
results.

\begin{enumerate}
\def\labelenumi{\arabic{enumi}.}
\setcounter{enumi}{3}
\tightlist
\item
  Test model without diameter parameter.
\end{enumerate}

\begin{Shaded}
\begin{Highlighting}[]
\NormalTok{lm.coefficients }\OtherTok{\textless{}{-}}\NormalTok{ lm.no\_diameter}\SpecialCharTok{$}\NormalTok{coefficients}

\NormalTok{prediction }\OtherTok{\textless{}{-}} \FunctionTok{calculate\_value}\NormalTok{(}\AttributeTok{a =}\NormalTok{ lm.coefficients[}\DecValTok{1}\NormalTok{], }\AttributeTok{weight =}\NormalTok{ lm.coefficients[}\DecValTok{2}\NormalTok{], }\AttributeTok{sexi =}\NormalTok{ lm.coefficients[}\DecValTok{3}\NormalTok{], }\AttributeTok{sexm =}\NormalTok{ lm.coefficients[}\DecValTok{4}\NormalTok{], }\AttributeTok{length =}\NormalTok{ lm.coefficients[}\DecValTok{5}\NormalTok{], }\AttributeTok{height =}\NormalTok{ lm.coefficients[}\DecValTok{6}\NormalTok{], }\AttributeTok{values =}\NormalTok{ data.test)}
\FunctionTok{plot}\NormalTok{(}\FunctionTok{sort}\NormalTok{(prediction))}
\FunctionTok{lines}\NormalTok{(data.test}\SpecialCharTok{$}\NormalTok{Age, }\AttributeTok{col =} \StringTok{"red"}\NormalTok{, }\AttributeTok{lwd =} \DecValTok{5}\NormalTok{)}
\end{Highlighting}
\end{Shaded}

\includegraphics{crabcrab_files/figure-latex/unnamed-chunk-16-1.pdf}

\begin{Shaded}
\begin{Highlighting}[]
\FunctionTok{compare\_to\_actual}\NormalTok{(prediction, data.test, }\DecValTok{8}\NormalTok{, }\DecValTok{8}\NormalTok{)}
\end{Highlighting}
\end{Shaded}

\begin{verbatim}
## [1] 0.8092974
\end{verbatim}

From this test we can see that probably diameter is not that important
either, as the prediction does not changes strongly, so we can ignore
it.

\begin{enumerate}
\def\labelenumi{\arabic{enumi}.}
\setcounter{enumi}{4}
\tightlist
\item
  Test model without diameter and height parameters.
\end{enumerate}

\begin{Shaded}
\begin{Highlighting}[]
\NormalTok{lm.coefficients }\OtherTok{\textless{}{-}}\NormalTok{ lm.no\_diameter\_height}\SpecialCharTok{$}\NormalTok{coefficients}

\NormalTok{prediction }\OtherTok{\textless{}{-}} \FunctionTok{calculate\_value}\NormalTok{(}\AttributeTok{a =}\NormalTok{ lm.coefficients[}\DecValTok{1}\NormalTok{], }\AttributeTok{weight =}\NormalTok{ lm.coefficients[}\DecValTok{2}\NormalTok{], }\AttributeTok{sexi =}\NormalTok{ lm.coefficients[}\DecValTok{3}\NormalTok{], }\AttributeTok{sexm =}\NormalTok{ lm.coefficients[}\DecValTok{4}\NormalTok{], }\AttributeTok{length =}\NormalTok{ lm.coefficients[}\DecValTok{5}\NormalTok{], }\AttributeTok{values =}\NormalTok{ data.test)}
\FunctionTok{plot}\NormalTok{(}\FunctionTok{sort}\NormalTok{(prediction))}
\FunctionTok{lines}\NormalTok{(data.test}\SpecialCharTok{$}\NormalTok{Age, }\AttributeTok{col =} \StringTok{"red"}\NormalTok{, }\AttributeTok{lwd =} \DecValTok{5}\NormalTok{)}
\end{Highlighting}
\end{Shaded}

\includegraphics{crabcrab_files/figure-latex/unnamed-chunk-17-1.pdf}

\begin{Shaded}
\begin{Highlighting}[]
\FunctionTok{compare\_to\_actual}\NormalTok{(prediction, data.test, }\DecValTok{8}\NormalTok{, }\DecValTok{8}\NormalTok{)}
\end{Highlighting}
\end{Shaded}

\begin{verbatim}
## [1] 0.8087691
\end{verbatim}

Finally, the last test we run on model without diameter and height, and
we can conclude that they are not very important parameters to consider
in harvesting crabs.

\hypertarget{conclusion}{%
\subsubsection{Conclusion}\label{conclusion}}

In this research we built and tested our linear regression models on
different sets of data. We discovered that for harvesting crabs
effectively farmers need to consider many of their physical attributes,
but to minimize the effort they can neglect the diameter and height and
focus on parameters which are more important in predicting their age -
length, weight, and sex of the crab.

\end{document}
